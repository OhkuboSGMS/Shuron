{\huge \textbf{論文概要}  \newline}
%\medskip

 {\Large \bf 研究の目的 \newline}

近年におけるニューラルネットワークを用いた深層学習による課題解決手法の発展によって
深層学習をゲームレベル作成に応用する研究が行われている.

機械学習ではデータセットの量が重要となるが既存のゲームレベルにおいては
100レベル程度であり、より多くのゲームレベルを準備できる環境を
整えることが必要である.

本研究はゲームレベルを機械学習から生成に
必要なデータセットが既存の環境では用意されていないため、
生成に必要な十分な量のレベルデータセットを用意できる環境を提案し、
その環境を実現するための実装を行うことが目的である,

\vspace{1cm}

{\Large \bf 論文全体のあらまし \newline}

既存環境ではゲームごとに与えられるレベル数は5レベルで
レベル表現方法も学習に使うためには修正をする必要があった.
このため,本研究においてはある1つのゲームから100万レベル以上の
レベルを抽出,解析,視覚化,挿入を可能にし,機械学習に
使いやすいようレベルを変換できる環境を実装した.
機械学習に使いやすいデータセットは実際に使用されていて
定義されていない手法をデータフォーマットとして定義し
変換できるようにする工夫をした.

\vspace{1cm}

{\Large \bf 各章の内容 \newline}

第\ref{introduction}章では本研究の研究背景,研究目的,論文構成について述べる.
第\ref{preparation}章では本論文を読む上で必要となる事前知識について述べる.
第\ref{conventional_method}章では 従来のレベル生成環境について述べる.
第\ref{proposed_method}章では 提案する環境 の実装方法について述べる.
第\ref{evaluation}章では 提案環境の動作確認と考察について述べる.
第\ref{conclusion}章では本論文の結論と今後の展望について述べる.

\vspace{1cm}

{\Large \bf 結論 \newline}

本研究では,スーパーマリオメーカ−2を対象としてゲームレベルを
抽出,挿入,編集できるよう実装することで機械学習によるレベルの生成
できる環境を構築した.動作確認としてGANを用いてレベル学習を行い,生成した
レベルを実際にゲーム上に反映することができた.

\vspace{1cm}
{\Large \bf 得られた成果の意義 \newline}

本研究では,スーパーマリオメーカ−2を対象として機械学習によるレベル生成のための
環境を実装した.これにより,データから学習した
モデルから新しいステージの生成が期待できる.
レベルデザイナーのレベル生成の負担削減のための生成モデルを作成するために
有益である.


